\section{Вывод}

В рамках данной работы было разработано и реализовано сетевое приложение на языке Java с использованием технологии EJB. Считаю, что приложение
разработано и реализовано на высоком уровне. Развитие приложения с функциональной точки зрения может быть продолжено в добавлении новых
бизнес процессов. С точки зрения развертывания, в первую очередь необходимо добавить возможность конфигурации приложения из конфигурационного
файла. В данном случае этого не было сделано по причине специфики работы выбранного сервера приложений с ресурсными файлами.

Дадим оценку приложения с точки зрения различных характеристик распределенных систем:
\begin{itemize}
    \item \textbf{Прозрачность} обеспечивается методикой REST на уровне представления приложения. 
    \item \textbf{Открытость} достигается тем, что приложение обладает сетевым интерфейсом. 
    \item \textbf{Масштабируемость} приложения оценить достаточно сложно. Можно только предположить, что достаточно разместить приложение 
        на распределенном сервере приложений.
\end{itemize} 